\section{Introduction}

Our company \moldco experiance currently a class action suit against us by an ex-employees. They are claiming that they are suffering from a noise induced hearing loss from working in our plants. 
According to the paper \emph{Noise induced hearing loss in china: A potentially costly public health lssue} by \emph{Shi Yongbing} and \emph{William Hal Martin} it's a recurrent problem in china and must be treated seriously.

In this document we decribe the action to plan during this crisis and how to communicate outside the company in order to avoid any bad opinions about \moldco.

\section{Action plan}

In order to act for this crisis, we need to follow this action plan that describe what to do to resolve at best this crisis.

\subsection{Prepare an applogize communication}

\moldco must applogize for injury of it's ex-employees.
This applogoize must be sincere and truthfully.
The language elements sould be presented to each pearson that could communicate or interviewed about this crisis, that include communication pole, CEO and communication chief.

We recommand the following paragraph of aplogize.

\bigskip

\emph{We applogize for all the inconvenients and inury that our ex-employees could suffer during his labour in \moldco and we're doing all web can do to reduce this risks in the future.}

\bigskip

\subsection{Discussion with ex-employees}

Our communication director must contact the ex-employees to offer them a good and valid alternative solution to avoid expensive lawsuit action.
This alternative could be a full cover of health care fees and potential fiancial compensation.
Further discussions are to be expected by ex-employees in order, to negociate this financial compensation.

If this amicably solution is accepted by all parties,  We havn't to enage lawsuit and press release about this crisis we can skip the next parts and got to the part \ref{endpart}.

In the other case, we have to engage a full press release and public communication about this crisis.

\subsection{Reassure and Manage}

This part is to engage in the next hour of a public release of this crisis.
During this part, \moldco must communicate to the press and shareholders.
A press release must be communicated to the press and all communication actors of \moldco, that include \moldco CEO and all communication pole.
You could found our proposal of press release in section \ref{press-release}.

Our press release have to reassure shareholders and public about \moldco to keep our brand reputation.

\subsection{Find solutions}
\label{endpart}

After, communication to the public, \moldco must take actions to correct this kind of problem in the company.
The company could call for external experts for council about sound problem in assembly lines.

In this case, actions to take could be :

\begin{itemize}
\item Install sounds sensors to find where sound level exceed the limit of 85 dB
\item Install sound reduction foam on machines to reduce vibrations and sound level
\item Give to employees Personnal Protection Equipment to protect theire ears
\end{itemize}

\moldco must take a lawyer to manage our class action suit.
Give him all information we have so far about out ex-employees and our press communication and apologize.

\subsection{Communicate and resolve}

The final part closes the crisis with a communication of actions took by \moldco to shareholders.
This communication must reassure all shareholders a potental investors or clients that none of this type of problem will happen again at \moldco.

\subsection{Archiving}

After all shareholders are aware of action took by \moldco and the class action suit is over, a report of all the event related to this crisis must be redacted and archived.
This report will help future crisis management and improvemet in similar crisis that can happen in the future.

\section{Press release}
\label{press-release}
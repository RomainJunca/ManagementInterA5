\documentclass{article}

\usepackage[utf8]{inputenc}
\usepackage[T1]{fontenc}
\usepackage[francais]{babel}
\usepackage[top = 1.5cm, left = 1.5cm, right =1.5cm, bottom = 1.5cm]{geometry}
\usepackage[pdfborder ={0 0 0}]{hyperref}
\usepackage{graphicx}
\usepackage{xcolor}
\usepackage{multicol}
\usepackage{lscape}
\usepackage{datatool}
\usepackage{pdfpages}
\usepackage{rotating}
\usepackage{xspace}
\usepackage{titlesec}
\usepackage{colortbl}
\usepackage{roboto}

\definecolor{title-color}{gray}{0.30}
\definecolor{heading-color}{gray}{0.90}

\setlength{\parindent}{15pt}
\setlength{\parskip}{5pt}

\titleformat{\section}
  {\normalfont\sffamily\Huge\bfseries\centering\robotoslab\color{title-color}}
  {}{0pt}{\bigskip}

\titleformat{\subsection}
  {\normalfont\sffamily\LARGE\bfseries\robotoslab\color{title-color}}
  {\thesubsection}{1em}{}

\titleformat{\subsubsection}
  {\normalfont\sffamily\large\bfseries\robotoslab\color{title-color}}
  {\thesubsubsection}{1em}{}

\begin{document}

\begin{titlepage}
\begin{figure}
\end{figure}

\title{\vspace{1cm}{\Huge \bf{\color{title-color}\robotoslab Synthèse de comportement professionnel en Chine} } \\ \vspace{2cm} \bf{\color{title-color}\robotoslab Travailler avec une entreprise Chinoise} \vspace{1cm} \\
}
\begin{figure}
\end{figure}
\author{\Large{Marie \bsc{Chiaverini}} \Large{Baptiste  \bsc{Saclier}} \\\Large{Vadim  \bsc{Crochet}} \Large{Antoine  \bsc{Caillet}} \\\Large{Romain  \bsc{Junca}}}
\date{}
\vfill 
\end{titlepage}
\maketitle
\vspace{\fill}
\Large{CESI school of engineers} \hfill \Large{Tutor : Thierry \bsc{BLANC}}
\thispagestyle{empty}
\setcounter{page}{0}
\newpage

\renewcommand{\contentsname}{Table of contents}
\tableofcontents

\newpage

\newcommand{\projectname}{ChineseTooth\xspace}
\newcommand{\companyname}{Cesi conseil\xspace}
\newcommand{\moldco}{MOLD \& Co.\xspace}

\section{Introduction}
Il est de plus en plus courant aujourd'hui pour les entreprises d'étendre leurs activités à l'international. Ainsi bien souvent, nous sommes amenés à nous retrouver, lors d'un projet, avec un interlocuteur étranger, qu'il soit client ou collaborateur.
Dans ce genre de situation, nos méthodes de travail ainsi que nos comportements en entreprises peuvent sembler étranges et même différer fondamentalement selon le pays.
En informatique notamment, il n'est pas rare de se trouver dans cette situation lors d'un projet. Le chef de projet doit donc faire face non seulement à la barrière de la langue, mais également celle de la culture.

Dans le cas particulier d'un projet en collaboration avec une entreprise Chinoise, il faut notamment comprendre la mentalité du pays pour adopter un comportement adéquat. Au travail, les comportements entre Français et Chinois peuvent être radicalement différents. \\


\section{Principales différences}
La culture Chinoise induit une pudeur et une loyauté très importante. En entreprise, l'autorité assez présente, est donc respectée, et ce même s'il arrive régulièrement que les Chinois travaillent entre proches ou personnes de la même famille. La pudeur impose de ne pas trop montrer ses sentiments, par peur d'imposer une opinion. Ce qui peut donc être vu comme de l'hypocrisie pour un travailleur Français et en fait considéré comme une forme de respect pour un travailleur Chinois.
Comme en France, le management est basé sur l'autorité hiérarchique, les Chinois ont un profond respect pour leurs responsables, qu'ils saluent d'abord en premier, avant de saluer les collègues. Il est impensable de montrer son désaccord face à ce dernier ! On tend à favoriser la politesse hiérarchique plutôt que général.

Souvent sous pression, le travailleur Chinois est valorisé pour son travail, une relation de confiance n'est possible avec ses collègues ou son responsable que selon son efficacité. Lors de réunions de travail, un "oui" de la part d'un collaborateur ne concernera pas les décisions ou le sujet, mais simplement le fait qu'il a bien compris ce qui est dit, il ne dira jamais "non" même pour reconnaître un échec.

Cependant, un échec ou une erreur de la part d'un employé Chinois ne sera pas vu aussi négativement qu'en France. Une erreur ne représente pas une finalité négative. L'idée générale est d'"apprendre en faisant", quitte à se tromper et changer sa méthode de travail. Mais il faut absolument tirer les leçons et prendre le temps de réfléchir aux améliorations. Un Français ou un Européen aura en revanche tendance à prendre un temps de réflexion avant d'être sûr d'être capable d'effectuer une tâche.

Cela s'inscrit dans l'"optimisme" Chinois qui indique que dans le mal, il y a du bien, et inversement dans le bien, il y a du mal. Il y a toujours un compromis ou un équilibre à trouver dans toute situation.

La notion de travail est donc très importante. Comme dit précédemment, la vie professionnelle prend le dessus sur la vie privée, si bien que beaucoup de Chinois décident de mélanger les deux. Ainsi, le lieu de travail devient progressivement un lieu de vie, où l'on vient en chausson, on se met à l'aise, et où l'on s'autorise même une sieste en début d'après-midi. \\


\section{Conseils sur le comportement}
En tant que Français, il est donc nécessaire de prendre en compte ces comportements lors d'un travail de collaboration avec des Chinois.

Premièrement, toujours dans la pudeur et la vision du respect des Chinois, les poignées de main, et évidemment les embrassades, sont à éviter absolument. Un salut rapide sans contact physique sera plus apprécié.

Les Chinois aiment avoir l'impression que les gens sont stables, ce qui aide dans la valorisation du travail et contribue à l'instauration d'une relation de confiance. C'est pour cette raison qu'il ne faut jamais se laisser impressionner ou déstabiliser. Il faut toujours garder son sang-froid (et le sourire) même dans une situation délicate. Lors d'une réunion ou d'un débat, il est important de montrer que nous sommes à l'écoute du collaborateur, en prenant soin de saluer la qualité de son travail, contribuant à une mise en confiance. De même que lors d'un échange, il est préférable d'éviter de faire des gestes, ceci étant vu comme signifiant que vous êtes en colère.

Il est également bien vu de faire des petits cadeaux à ses responsables ou collaborateurs. Symboliquement, les Chinois apprécient beaucoup les présents. Une bouteille de champagne ou une boîte de chocolat peuvent contribuer à renforcer les relations au sein d'une équipe de travail.

Justement, lors d'un travail en équipe, la rapidité est importante. Toujours dans l'optique d'"apprendre en faisant", un temps de réflexion trop long peut être considéré comme une perte de temps. On vient d'ailleurs travailler tôt le matin, avec un déjeuner généralement entre 11h et 11h30.\newpage

Pour résumer, c'est en respectant la pudeur et l'importance du travail dans les relations avec les autres que nous serions valorisés dans un projet incluant des collaborateurs chinois. En prenant en compte les valeurs de leur culture et leurs habitudes en entreprise, une relation de confiance s'installera rapidement et amènera à de bonnes conditions de travail dans un projet interculturel. \\


\end{document}


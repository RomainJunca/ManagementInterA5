Il est important de mesurer de façon continue l'évolution du projet. C'est pour cela que la section qui va suivre traitera des indicateurs clés de performance.

La représentation de tous les indicateurs choisis sont inscrits dans un tableau où l'on retrouve : 
\begin{itemize}
    \item le nom de chaque indicateur,
    \item l'objectif défini au début du projet,
    \item la valeur réelle de l'indicateur.
\end{itemize}

Si l'indicateur est a vert, il faudra poursuivre les actions en cours afin de maintenir ce bon résultat.
Si l'indicateur est au rouge, vous devez prendre les mesures correctives nécessaires.
Si l'indicateur est au orange, il faut alors le surveiller.

Les indicateurs clés de performance sont à suivres de près, en effet, ils ont d'une certaine façon, un impact financier sur le projet. Si l'indicateur est au rouge, les mesures correctives qui s'imposent généreront des dépenses additionnelles. Toutefois, si l'indicateur est au vert, c'est que tout se déroule comme prévu.

Nous classerons nos indicateurs clés de performance sous quatre catégoris : 
\begin{description}
    \item[Les délais]: le projet se déroule dans les temps
    \item[Le budget]: budget dépassé ? 
    \item[La qualité] : la progression du projet est-elle satisfaisante ?
    \item[L'efficacité] : Le projet est-il gérer de manière efficace ?
\end{description}



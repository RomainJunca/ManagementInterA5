\documentclass{article}

\usepackage[utf8]{inputenc}
\usepackage[T1]{fontenc}
\usepackage[francais]{babel}
\usepackage[top = 1.5cm, left = 1.5cm, right =1.5cm, bottom = 1.5cm]{geometry}
\usepackage[pdfborder ={0 0 0}]{hyperref}
\usepackage{graphicx}
\usepackage{xcolor}
\usepackage{multicol}
\usepackage{lscape}
\usepackage{datatool}
\usepackage{pdfpages}
\usepackage{rotating}
\usepackage{xspace}
\usepackage{titlesec}
\usepackage{colortbl}
\usepackage{roboto}
\usepackage{pdfpages}
\usepackage{lscape}
\usepackage{longtable}

\definecolor{title-color}{gray}{0.30}
\definecolor{heading-color}{gray}{0.90}

\setlength{\parindent}{15pt}
\setlength{\parskip}{5pt}

\titleformat{\section}
  {\clearpage\normalfont\sffamily\Huge\bfseries\centering\robotoslab\color{title-color}}
  {}{0pt}{\bigskip}

\titleformat{\subsection}
  {\normalfont\sffamily\LARGE\bfseries\robotoslab\color{title-color}}
  {\thesubsection}{1em}{}

\titleformat{\subsubsection}
  {\normalfont\sffamily\large\bfseries\robotoslab\color{title-color}}
  {\thesubsubsection}{1em}{}

\begin{document}

\begin{titlepage}
\begin{figure}
\end{figure}

\title{\vspace{1cm}{\Huge \bf{\color{title-color}\robotoslab Management Plan} } \\ \vspace{2cm} \bf{\color{title-color}\robotoslab Mold \& Co in China} \vspace{1cm} \\
}
\begin{figure}
\begin{center}
\includegraphics[scale = 0.2]{Img/china-illustration.jpg}
\end{center}
\end{figure}
\author{\Large{Marie \bsc{Chiaverini}} \Large{Baptiste  \bsc{Saclier}} \\\Large{Vadim  \bsc{Crochet}} \Large{Antoine  \bsc{Caillet}} \\\Large{Romain  \bsc{Junca}}}
\date{}
\vfill 
\end{titlepage}
\maketitle
\vspace{\fill}
\Large{CESI school of engineers} \hfill \Large{Tutor : Thierry \bsc{BLANC}}
\thispagestyle{empty}
\setcounter{page}{0}
\newpage

\renewcommand{\contentsname}{Table of contents}
\tableofcontents

\newpage
\renewcommand{\listfigurename}{List of figures}
\listoffigures

\newpage

\newcommand{\projectname}{ChineseTooth\xspace}
\newcommand{\companyname}{Cesi conseil\xspace}
\newcommand{\moldco}{MOLD \& Co.\xspace}

\section{Introduction}

This document describes all aspects of the \projectname project which the main goal is to install IT systems around the new production line in the eco-city of Taijin.
This project includes a social and ecological aspect in order to fit to the requirements of Taijin city guidelines.

In this document, we describe what are the goals, the processes, the planning and the risk of such deployment in China.

\section{Project description}

% Description précise de tout les aspects du projet
% Quel sont les objectifs ? 
% Dans quel contexte le projet évolue il ?
% Quels sont les contraintes (Environnement, Social, Temps, Législation) ?
% Quels sont les forces et les faiblesses ?

The main goal is to install a toothbrush production line in the eco-city of Taijin in China.
Our company is workig for \moldco to make this production line a reality.

Ou main guidelines in this project is to install a production line that can produce a great amount of toothbrushes within an eco-city.
This project needs to be respectful of the surrounding environnement and social aspects of the project's stakeholders.

\subsection{Specifications}

This project has to achieve the following specifications.

The production line must contains all the required machines to automate the production of toothbrushes.
Theses machines include moulting machine, stamping machine, tufting machine, bristle cutter machine, bristle trimming machine and Packaging machine.
These machines need to be bought and connected to each other in order to build the full product.

To connect all the machines in the assembly line, the project requires also a full digital connection to an internal network.
This network group all connected machines and database servers to store monitoring informations about the production.
These informations need to represent the current production, the past production and potential errors in the production line.

The production line is fully automated throw this network and the production is regulated to produce exactly what is needed.
This automatisation brings many advantages including the environmental impact reduction, reduction of the storage requirement of finished products and 24/7 production in case of huge demand.

The informations collected need to be displayed to the employees in charge of the production line.
These informations are displayed throught an interface reading the monitoring data from the database.
A master server has to be installed in order to control all machines and to control the production flow.

Several materials are required to produce toothbrushes.
These materials are plastic, nylon, brass wire, paper box packing, plastinc hard container packaging, high frequency blister packaging and Blister card packaging.
The project must include a storage space for all these material and human resources to load the resources in the appropriate machines.

All the production line machines, storage and digital network requires engineering the organise all these components depending on the space available and the shape of the building.
Enginieering human resources are required to create, configure and install manitoring system.
Human resources may also be required to manipulate machines, connect each machine to the other and install network.

\subsection{Forces}

The forces of the project are mainly focused on the high effeciency of the production line.
This high effeciency is garanteed by the monitoring system and the automatic management of the amount of product produced on the assembly line.
This project represents a great opportunity to modernize the production of \moldco and automate the assembly line.
By automating the assembly line, \moldco gain a lot of money on storage of manufatured products and human resources.

\subsection{Weaknesses}

This project have also small weakness that may have an impact on risks
(Risk managment will be covered in the section \ref{risk-management}).

The main weakness are the important amount of advanced technologies that requires a great amount of high qualified employees in charge of the installing and maintaining the autonomous system of the assembly line.
Another weakness is the requirement of heavy and pricy machines that can represent a major part of the project's costs.

\section{Actors and Stakeholders}

% Quels sont les différents acteurs ayant un impact sur le projet ?
% Quels sont les parties prenantes et leur position dans le projet ?
% Quels sont les équipes que l'on doit mettre en place ?

We have assembled all the actors of the project in a clear and precise way in order to identify them. You will first find the different actors who have an impact on the project. Secondly, the stakeholders and their position in the project. Finally, the teams that need to be set up.

\subsection{Actors impacting the project}

You will find below a table containing all the actors having an impact on the project. All the stakeholders were identified and analysed according to the client's needs by the \companyname team.

There are four columns :

\begin{description}
    \item[Name] : it is the name of the actor and stakeholder.
    \item[External or Internal to \moldco companie] : The actor in question is internal or external to \moldco. This is its positioning within the project.
    \item[State] : what type of domain is the actor affiliated.
    \item[Influence level] : this is the level of importance of the actor in the project.
\end{description} 

\begin{figure}[h]
\centering
\begin{tabular}{| p{4cm} | c | c | c |}
    \hline
    \rowcolor{heading-color}\multicolumn{1}{|c|}{Name} & External or internal & Status & Influence level \\
    \hline
    Mold and Co - HR department & Internal & Supervision & Important \\
    \hline
    Mold and Co - Production department & Internal & Manufacturation & Important \\
    \hline
    Mold and Co's direction & Internal & Client & Important \\
    \hline
    \companyname & External & Provider & Important \\
    \hline
    Mold and Co's - It department & Internal & Supervision & Medium \\
    \hline
    Mold and Co's - Maintenance department & Internal & Supervision & Medium \\
    \hline
    Mold and Co's - Logistic department & Internal & Supervision & Medium \\
    \hline
    Tianjin city hall & External & Notice of construction & Important \\
    \hline
    People's Republic of China government & External & Supervision & Important \\
    \hline
    Suppliers & External & Supply & Important \\
    \hline
\end{tabular}
\caption{Table of stakeholders}
\end{figure}


\subsection{Setting up teams}

Following the stakeholder analysis for this project, we set up teams to maximize the company's production and meet the Chinese company's standards.

These are three teams distributed as a service to ensure the proper functioning of the company Chinetooth.

\begin{figure}[h]
\centering
\begin{tabular}{| c | p{6cm} | c |}
    \hline
    \rowcolor{heading-color}Name & \multicolumn{1}{c|}{Objective} & influence level \\
    \hline
    Human Resource department & Recruit new employees, retain them and develop their skills. & Important \\
    \hline
    Engineering department & Conception, resource planning, scheduling, recording and traceability of production activites & Important \\
    \hline 
    Assembly line installation department & storage and installation of machines & Important \\
    \hline 
\end{tabular}
\caption{Table of teams working on the project}
\end{figure}

\begin{description}
    \item[Humain resource department]: will help to maintain a stable workforce over the long term.
    \item[Engineering department]: its objective is to continuously improve the management of flows and stocks included in the work chain that begins with suppliers and ends with intermediate or end customers. There are three engineer department, one for machines, second for network and the last for industry 5.0.
    \item[Assembly line installation department]: the role of the marketing department is to define a company's strategy by proposing products and services that will promote the development and sustainability of Mold \& Co. There are three teams, one for resource installation, second for network installation and the last for IoT installation.
\end{description} 


\section{Project planning}

% Liste des différentes taches à éfféctuer pour atteindre les objectifs 
% Planning prévisionnel des taches
% Deadlines
% Association des équipes à chaque tache
% Quels sont les marges ? Le chemin critique ?

\begin{figure}[h]

\centering
\includegraphics[scale=0.5]{Img/wbs-management-inter.pdf}
\caption{Work Breakdown Structure of the project}

\end{figure}

\subsection{Tasks}

The project is separated into many tasks that represent all steps needed to reach the goal of the project.
These tasks are separated in two categories: \emph{Management plan} in which all the tasks represents the redaction od the management plan of the project and \emph{Executive plan} the represent the active part of the project in which the assembly line is installed.

\subsubsection{Management Plan}

The management plan is the part in which each step of the project is defined.
The management plan is defined as a frame for the project and theses tasks must be achieved before all executive tasks.
This part begins with a precise description of the project and goals followed by the 5 next parts.

\paragraph{Actors and Steakholders} These parts, we must think and describe all the actors involved in the project.
These actors are stakeholders and can interact in some way with the project.
The description includes their position, their importance and the manner that they interact with the project.
This task has also a goal of definition different teams that are required to bring this project to life.

\paragraph{Project planning} Within this task, we must think and describe all the tasks required to finish the full project, the time and resources required to achieve each task and what are task dependencies.
In this task, we must define what are deadlines and when to make debriefing and evaluate the progression of the project.

\paragraph{Resources} In this part of the project, we must identify and write the required resources to achieve each task defined in the \emph{Project planning} part.
These resources include human resources, financial resources, and material resources.

\paragraph{Risks} In this task, we must define the primary risks that can occur during the project and how to reduce the side effect of each risk.
Each risk has a severity and probabilty rate that represent the criticality of it.
Higher the criticism is, important the risk must be and planned with caution.

\paragraph{Progression and success monitoring} Finally, in the progression monitoring, we must identify what indicator can represent the progression or the success of each task and the whole project.
These indicators will be used all along the project to define its progression and if some tasks are taking late.

\subsubsection{Executive Plan}

The executive plan indicates all actions done after the work on the management plan. 
This category includes the installation of the assembly line and its automation, but also it is control and monitoring, besides the human resources and an audit of the client.

\paragraph{Assembly Line Installation} This part aims to identify the processes brought by the installation of the assembly line. 
After an assembly line engineering, in which we study the building disposal, where and how the machines will connect with themselves, we also
study the place for the storage areas. 
We then test the machines after their purchase and their shipping. 

\paragraph{Assembly Line Automation installation} This part is about the automation of the assembly line, which includes a network engineering (the study of the disposal of the network in the building), the purchase
of the material for this network (routers, switches, etc.) and their shipping to the building, before their installation and test. 
We also study the automation of the machines, how it will work, and how to put it in place, with also a testing session and a security check.

\paragraph{Human Resources} The human resources part identifies the employee selection process. Those employees 
must be fit to the required tasks of the executive plan. To the study of the building and another engineering around the machines to the installation of the automation of those machines, and their control and monitoring. 

\paragraph{Control and Monitoring} After the installation of the machines and their automation set, we must control and regulate them.
This includes the risks analysis process, which means a constant control and verification of the assumed risks but also an answer plan in the case of a crisis. 
We also check on the quality of the machines, their cleanliness and their working order, but also the employee's satisfaction as we want to be sure they work in an environment as comfortable as possible. 
In the same way, we want a constant control of the ecological footprint of the building to respect our environmental engagements.   

\paragraph{Audit} Finally, this task is required to retrieve some feedback from the client.
This feedbacks can lead to an improvement process and can be added to our quality pipeline in order to continuously improve our practices.


\subsection{Gantt diagram}

The Gantt diagram of this project recapitulating all our planning is in the first Annex of this document. This is an export of Microsoft Project's file of this project.


\section{Required resources}

% Quels sont les moyens humains nécéssaires ?
    % Equipes
    % Formations
    % Nombres d'heures
% Quels sont les moyens financiers requis ?
% Quels sont les moyens materiel requis ?

%\subsection{Human cost}
The human cost we analyze represents the amount of work in days and the remuneration of everyone involved in the realization of, based on our project planning, the two parts (the initial part with the management plan, and the executive plan with the installation and monitoring of one assembly line, and the formation of the employees to the handling of the machines).
The costs are based on the total number of days of work,the persons involved, and the average price cost of the employees which you can see in this table.

\begin{figure}[h]

\centering
\includegraphics[scale=1]{Img/humanCost.png}
\caption{Average employee cost}

\end{figure}

To calculate the cost we used the working rules in France and China :
\begin{itemize}
	\item[--] In France : 35 hours of work per week and 272 days of work per year.
	\item[--] In China :  40 hours of work per week and an average of 345 days of work per year. \\
\end{itemize}

The next table will present the different types of actors (employees) of this project and how much they are paid per hour according to the average employee cost of the previous table and the working rules for each country.

\begin{figure}[h]

\centering
\includegraphics[scale=0.6]{Img/HumanCostPerHours.png}
\caption{Average employee costs per hour}

\end{figure}

With this data, we now analyze the number and type of the actors of each part and calculate the total cost.

\subsubsection{Initial part}
	The initial part took place for 4 days and involved a team of five french engineers, one leader senior engineer and four junior engineers.
	The cost is calculated according to the french working rule of 35 hours of work a week, so 7 hours a day.

	\begin{figure}[h]

	\centering
	\includegraphics[scale=0.6]{Img/firstPartHumanCost.png}
	\caption{Initial part cost}

	\end{figure}

\subsubsection{Executive part}
	According to the Gantt diagram, the executive part is 102 days long, minus 80 days of equipment shipping (twice 40 days), it represents 22 days of work.
	During these 22 days, 7 Chineses engineers and 10 Chineses technicians/regular employees (counted as technicians), as you can see here, we estimated these numbers while partitioning the executive part (installation and monitoring/control) in different tasks.\\

	\begin{figure}[h]

	\centering
	\includegraphics[scale=0.6]{Img/secondPartNbrHuman.png}
	\caption{Number of employees for the executive part}

	\end{figure}

	The cost is calculated according to the Chinese working rule of an average of 40 hours of work a week, so 8 hours a day.

	\begin{figure}[h]

	\centering
	\includegraphics[scale=0.6]{Img/secondPartHumanCost.png}
	\caption{Executive part cost}

	\end{figure}

\subsubsection{Formation}
	The 17 employees involved in the executive part need to be tought how an assembly line works, how to handle the machines, how they work and also their security rules. 

	We estimated the cost of such a formation of 3000 euros per person (a total of 51 000 euros for the 17 employees) and 5 days, also counted as 5 days of works for them.

	\begin{figure}[h]

	\centering
	\includegraphics[scale=0.6]{Img/formationCost.png}
	\caption{Formation cost}

	\end{figure}

With these three parts, we reach a total human cost of \textbf{84 626.96 euros.} \\

\subsection{Material cost}
This cost is about every material directly used in one assembly line and its cost.
Our assembly line will include : \\
\begin{itemize}
	\item[--] Handle molds, to make the brush handle (an average of 2 per injection machine).
	\item[--] An injection machine, to mold the shape of the toothbrushes.
	\item[--] A tufting machine to tuft on brush holders.
	\item[--] A trimming and end rounding machine to cut and shape the bristles to the manufacturers specification, and to round them to be softer and more comfortable to the teeth.
	\item[--] A fully automated packaging machine to pack the toothbrushes.
	\item[--] And of course conveyers belt which will link these machines together. We estimated an average of 4 meters between machines, so we would need around 16 meters of it. \\
\end{itemize}

All of it would be around 30 square meters. \\
We have access to two types of injection machines, a 50T and a 80T, which means it is a 50/80 ton servo-motor operated machine, the maximum clamping force with these machines is either 50 or 80 tons. Servo-motors are used for energy saving, so these machines give the highest energy saving in hydraulic machines. In this estimation we chose the 80T injection machine for a better energy saving.

To calculate the total cost, we used this table of average prices. \clearpage

\begin{figure}[h]

	\centering
	\includegraphics[scale=0.9]{Img/averageMachineCost.png}
	\caption{Average machine costs}

\end{figure}

We have 5 machines in our assembly line. We decided, for the electrical/hydraulic/water costs, that there will be a maximum of 3 machines by post, so 2 posts for one assembly line, which will cost 6 000 euros.

\begin{figure}[h]

	\centering
	\includegraphics[scale=0.6]{Img/machineCost.png}
	\caption{Machines costs for one assembly line}

\end{figure}

So, the total machine cost for one assembly line is \textbf{188 000 euros}.

We reach a total estimation cost of \textbf{272 626.96 euros} with both the human cost and the machine cost.


\section{Risks management plan}
\label{risk-management}
In order to deal with the risk management, we decided to use the FMECA (Failure Mode, Effects, Criticality, Analysis) Method on the means of production of\moldco company. This means that we consider all the risks related to the operation of the assembly line system.\\

For that purpose, we have, to as a first step, to describe the system. Then we will be able to identify all the threats. After this step, we will identify all the threats that could affects the system and assess them in order to define their criticality.\\
Ultimately, we will see how to compensate these eventual failure modes where appropriate.\\

\subsection{Description of the system}

The assembly line system is composed of two main parts : 
\begin{itemize}
    \item the mechanical system
    \item and the computing's one.\\
\end{itemize}

By the way, we estimate that these two systems are interdependent. In orther words, we have to take into account that if the mechanical part of the assembly line fails, the entire system cannot work anymoer and vice versa.\\
Actually, mechanical part of the production chain can work independently but if the computing system is not working, the mechanical will get problems about its regulation and monitoring.\\

\begin{figure}[h]
    \centering
    \begin{tabular}{| p{4cm} | c | c | c | c | c | c | c | c | c |}
        \hline
        \rowcolor{heading-color}\multicolumn{1}{|c|}{Failure mode} & Failure causes & Failure effect & Detection method & Corrective actions & Severity & Occurrence & Detection & Criticality & RPN\\
        \hline
        Power outage & ? & Stop the production & Beep & Beep & 5 & 1 & 1 & 5 & 5 \\
        \hline
        Noise pollution & Mechanical disruption & Employee's discomfort & Error message & Automatic measurement & 1 & 2 & 2 & 2 & 4 \\
        \hline
        Inherent materiel defect & Obsolescence & Slowdown or stop the production & Error message & Periodic maintenance & 5 & 2 & 1 & 10 & 10 \\
        \hline
        Earthquake & Environment & Stop the production &  &  &  &  &  &  &  \\
        \hline
        Fire & Overheat & Stop the production &  &  &  &  &  &  &  \\
        \hline
        Flood &  & Stop the production &  &  &  &  &  &  &  \\
        \hline
    \end{tabular}
    \caption{FMECA Method}
    \end{figure}

According to the FMECA MEthod, we have to assess two different values in the table of risk.\\

First of all, the Severity (S) which equals to the importance of the consequences that the risk could induce on the production of the factory. More the risk can affect the production line, more the Severity level is high as the following table shows :

\begin{figure}[h]
    \centering
    \begin{tabular}{| p{4cm} | c | c |}
        \hline
        \rowcolor{heading-color}\multicolumn{1}{|c|}{Severity definition} & Severity level & Associated color\\
        \hline
        Insignificant & 1 & green  \\
        \hline
        Minor & 2 & green clair  \\
        \hline
        Significant & 3 & yellow  \\
        \hline
        Serious & 4 & orange  \\
        \hline
        Major & 5 & red  \\
        \hline
    \end{tabular}
    \caption{Table of severity level}
\end{figure}

    Regarding the Occurrence (O), it enables to measure the likelihood of the risk. Mire the risk is potential, more the Occurrence value is high as we can see as below :

    \begin{figure}[h]
        \centering
        \begin{tabular}{| p{4cm} | c | c |}
            \hline
            \rowcolor{heading-color}\multicolumn{1}{|c|}{Occurrence definition} & Occurrence level & Associated color\\
            \hline
            Remote & 1 & green  \\
            \hline
            Very low & 2 & green clair  \\
            \hline
            Low & 3 & yellow  \\
            \hline
            Moderate & 4 & orange  \\
            \hline
            Major & 5 & red  \\
            \hline
        \end{tabular}
        \caption{Table of occurence level}
\end{figure}

Finally, the Detection (D) is measured according to the ease for the operators to discover a failure mode. Higher the value is, easier it is to detect the failure.

\begin{figure}[h]
    \centering
    \begin{tabular}{| p{4cm} | c |}
        \hline
        \rowcolor{heading-color}\multicolumn{1}{|c|}{Detection definition} & Detection level\\
        \hline
        Blatant & 1  \\
        \hline
        Easily identifiable & 2  \\
        \hline
        Discreet & 3  \\
        \hline
        Hard to identify & 4 \\
        \hline
        Very  hard to identify & 5 \\
        \hline
    \end{tabular}
    \caption{Table of detection level}
\end{figure}

Severity, Occurrence and Detection are arbitrarily assessed whereas Criticality and RPN are calculated from these same data.\\

Especially, we have the following relations :
\begin{itemize}
    \item Criticality = Severity \* Occurrence
    \item RPN = Severity \* Occurrence \* Detection = Criticality \* Detection
\end{itemize}

These last two measures enable us to prioritize the risk in order to make the further effort in order to limit them and eventually resolve them if ever they occur.

\includegraphics{Img/img-risk.png}

\subsection{Identification of the risks}
\subsubsection{Categorization of the risk}
\subsection{Enhancement of the assembly line function}

% Quels sont les risques encourus durant le projet ?
% Quel est la sévérité, la probabilité d'apparition ?
% Quel opérations mettre en place pour les risques les plus graves ?

\section{Indicators of progression and success}

The planning and development of the installation of toothbrush production line in the eco-city of Taijin in China is guided by a comprehensive set of Key Performance Indicators (KPIs) covering its ecological, economic and social development.\\

There are seven quantitatives and three qualitatives KPIs.

\subsection{Quantitatives KPIs}

\subsubsection{Developing a Dynamic and Efficient Economy}
\begin{description}
    \item[Use of renewable resources]: using recycled resources to save money by 40\%.
    \item[Control the production]: Production control to avoid overproduction, which can be costly in terms of storage and resources.
    \item[Transportation]: Use intelligent way of transportation in order to save mony by 60\%.
    \item[Client satisfaction]: Gather user feedback to improve the product.
\end{description}

\subsubsection{Developing efficient machines}
\begin{description}
    \item[Maintenance machine]: machines must be operational at least 99\% of the time.
    \item[Cleaning machine]: the machines must not know any dirt that may impact the quality of the product.
\end{description}

\subsubsection{Developing efficient employers}
\begin{description}
    \item[Formation]: In order to improve the quality and productivity of employees by 70\%.
\end{description}

\includegraphics{Img/tableaudebord.png}

\subsubsection{Qualitative KPIs}
\begin{itemize}
    \item Maintain quality and safe production through careful monitoring of machines and production.
    \item Adopt safety policies for employees that will promote their well-being and the smooth running of production.
    \item Maintain the most eco-responsible production line by following the environmental standards in the factory.
\end{itemize}



% Comment peut on quantifier l'avancement du projet ?
% Quel est l'indicateur permettant de définir qu'une tache est accomplie et valide ?

\section{Conclusion} 

The company Mold \& Co called our company (Cesi conseil) to set up a 5.0 toothbrush production line. This production line is being set up in China, in Taijin. This city is known to be eco-responsible. It is therefore important to monitor social and ecological aspects in order to follow the city's standards.

 The implementation of this production line will allow the company to benefit from an increase in turnover. However, with the different aspects of the installation, risks can arise. It is important to note the risk analysis in this document.

 For the better of the organization of this project, the realization of a task schedule (WBS) was done.
As a result, an analysis of the project's stakeholders concluded with the creation of three teams.

 The implementation of KPI makes it possible to know the performance and success indicators of the implementation of the production line.

 Finally, thanks to the budget, our team studied the need for the necessary resources to carry out this project.

\section{Annex 1 : Project Gantt diagram}

You can found our Gantt project in the following pages.

\includepdf[pages=-]{gantt.pdf}

\end{document}



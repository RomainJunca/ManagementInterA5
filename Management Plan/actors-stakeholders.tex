We have assembled all the actors of the project in a clear and precise way in order to identify them. You will first find the different actors who have an impact on the project. Secondly, the stakeholders and their position in the project. Finally, the teams that need to be set up.

\subsection{Actors impacting the project}

You will find below a table containing all the actors having an impact on the project. All the stakeholders were identified and analysed according to the client's needs by the \companyname team.

There are four columns :

\begin{description}
    \item[Name] : it is the name of the actor and stakeholder.
    \item[External or Internal to \moldco companie] : The actor in question is internal or external to \moldco. This is its positioning within the project.
    \item[State] : what type of domain is the actor affiliated.
    \item[Influence level] : this is the level of importance of the actor in the project.
\end{description} 

\begin{figure}[h]
\centering
\begin{tabular}{| p{4cm} | c | c | c |}
    \hline
    \rowcolor{heading-color}\multicolumn{1}{|c|}{Name} & External or internal & Status & Influence level \\
    \hline
    Mold and Co - HR department & Internal & Supervision & Important \\
    \hline
    Mold and Co - Production department & Internal & Manufacturation & Important \\
    \hline
    Mold and Co's direction & Internal & Client & Important \\
    \hline
    \companyname & External & Provider & Important \\
    \hline
    Mold and Co's - It department & Internal & Supervision & Medium \\
    \hline
    Mold and Co's - Maintenance department & Internal & Supervision & Medium \\
    \hline
    Mold and Co's - Logistic department & Internal & Supervision & Medium \\
    \hline
    Tianjin city hall & External & Notice of construction & Important \\
    \hline
    People's Republic of China government & External & Supervision & Important \\
    \hline
    Suppliers & External & Supply & Important \\
    \hline
\end{tabular}
\caption{Table of stakeholders}
\end{figure}


\subsection{Setting up teams}

Following the stakeholder analysis for this project, we set up teams to maximize the company's production and meet the Chinese company's standards.

These are three teams distributed as a service to ensure the proper functioning of the company Chinetooth.

\begin{figure}[h]
\centering
\begin{tabular}{| c | p{6cm} | c |}
    \hline
    \rowcolor{heading-color}Name & \multicolumn{1}{c|}{Objective} & influence level \\
    \hline
    Human Resource department & Recruit new employees, retain them and develop their skills. & Important \\
    \hline
    Engineering department & Conception, resource planning, scheduling, recording and traceability of production activites & Important \\
    \hline 
    Assembly line installation department & storage and installation of machines & Important \\
    \hline 
\end{tabular}
\caption{Table of teams working on the project}
\end{figure}

\begin{description}
    \item[Humain resource department]: will help to maintain a stable workforce over the long term.
    \item[Engineering department]: its objective is to continuously improve the management of flows and stocks included in the work chain that begins with suppliers and ends with intermediate or end customers. There are three engineer department, one for machines, second for network and the last for industry 5.0.
    \item[Assembly line installation department]: the role of the marketing department is to define a company's strategy by proposing products and services that will promote the development and sustainability of Mold \& Co. There are three teams, one for resource installation, second for network installation and the last for IoT installation.
\end{description} 
